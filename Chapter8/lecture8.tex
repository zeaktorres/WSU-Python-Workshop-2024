\documentclass{beamer}
\newcommand\tab[1][1cm]{\hspace*{#1}}
\usepackage{listings}
\begin{document}
\title{Python Chapter 8: Heap Queues}
\author{Ezequiel Torres}
\date{\today}
\frame{\titlepage}
\frame{\frametitle{Table of contents}\tableofcontents}

\section{Heap}
\subsection{Basics of Heaps}
\frame{\frametitle{Basics of Heaps}
    Heaps are a binary tree where the root is always less than the
    children
    \vspace{\baselineskip}

    We can use this to sort a list, or to have a list where we know the top element is 
    always the smallest
}


\begin{frame}[fragile]
\subsection{Heap Example}
\frametitle{Heap Example}
\begin{lstlisting}
import heapq
h = []
heapq.heappush(h, 3)
heapq.heappush(h, 6)
heapq.heappush(h, 2)
heapq.heappush(h, 9)
print(h[0]) # Prints 2
\end{lstlisting}
\end{frame}

\begin{frame}[fragile]
\subsection{Heap Pop Example}
\frametitle{Heap Pop Example}
\begin{lstlisting}
import heapq
h = []
heapq.heappush(h, 3)
heapq.heappush(h, 6)
heapq.heappush(h, 2)
heapq.heappush(h, 9)
heapq.heappop(h)

print(h[0]) # Prints 3
\end{lstlisting}
\end{frame}

\begin{frame}[fragile]
\subsection{Heapify Example}
\frametitle{Heapify Example}
\begin{lstlisting}
import heapq
h = [3,6,2,9]
heapq.heapify(h)

print(h[0])
\end{lstlisting}
\end{frame}

\begin{frame}[fragile]
\subsection{Heap Sort}
\frametitle{Heap Sort}
\begin{lstlisting}
import heapq
h = []
for value in [3,6,2,9]:
    heapq.heappush(h, value)

# Prints 2,3,6,9
print([heapq.heappop(h) for i in range(len(h))]) 
\end{lstlisting}
\end{frame}
\end{document}
