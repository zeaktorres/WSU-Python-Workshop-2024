\documentclass{beamer}
\newcommand\tab[1][1cm]{\hspace*{#1}}
\usepackage{listings}
\begin{document}
\title{Python Chapter 7: Stacks and Queues using Dequeue}
\author{Ezequiel Torres}
\date{\today}
\frame{\titlepage}
\frame{\frametitle{Table of contents}\tableofcontents}

\section{Stacks}
\subsection{Basic Stack}
\frame{\frametitle{Basics of Stacks}
    When we want to utilize the full capabilities of a linked list
    we can use a Stack. Stacks are LIFO, Last in First out, meaning
    whoever joins the stack last is out first.
    \vspace{\baselineskip}

    When we want to grab the element from the top of the stack, we use
    .pop() and it returns the last element.
    \vspace{\baselineskip}

    For our implementation of a stack we will use the deque library from collections
}


\begin{frame}[fragile]
\subsection{Stack Example}
\frametitle{Stack Example}
\begin{lstlisting}
# Python code to demonstrate Implementing  
# Stack using deque 
from collections import deque 
queue = deque(["Ram", "Tarun", "Asif", "John"]) 
print(queue) 
queue.append("Akbar") 
print(queue) 
queue.append("Birbal") 
print(queue) 
print(queue.pop()) # Returns Birbal
print(queue.pop()) # Returns Akbar   
print(queue) 
\end{lstlisting}
\end{frame}

\section{Queues}
\subsection{Basics of Queues}
\frame{\frametitle{Basics of Queues}
    Queues are FIFO, First in First out, meaning
    whoever joins the queue first is out first. A
    great a analogy is thinking of a line to a movie, 
    whoever is in front of the line is helped first.
    \vspace{\baselineskip}

    For our implementation of a queue we will use the deque library from collections; however,
    we will use popLeft() to grab the first element
}

\begin{frame}[fragile]
\subsection{Queue Example}
\frametitle{Queue Example}
\begin{lstlisting}
# Python code to demonstrate Implementing  
# Stack using deque 
from collections import deque 
queue = deque(["Ram", "Tarun", "Asif", "John"]) 
print(queue) 
queue.append("Akbar") 
print(queue) 
queue.append("Birbal") 
print(queue) 
print(queue.popleft()) # Returns Ram
print(queue.popleft()) # Returns Tarun   
print(queue) 
\end{lstlisting}
\end{frame}


\begin{frame}[fragile]
\section{Dequeue Methods}
\frametitle{Dequeue Methods}
\begin{lstlisting}
# Python code to demonstrate Implementing  
# Stack using deque 
from collections import deque 
queue = deque(["Ram", "Tarun", "Asif", "John"]) 
queue.append("Akbar") 
queue.append("Birbal") 
queue[-1] # Returns the first last element in O(1)
queue[0]  # Returns the last element
queue     # Returns True if the list has elements
\end{lstlisting}
\end{frame}
\end{document}
