\documentclass{beamer}
\newcommand\tab[1][1cm]{\hspace*{#1}}
\usepackage{listings}
\begin{document}
\title{Python Chapter 2: Arrays and For Loops}
\author{Ezequiel Torres}
\date{\today}
\frame{\titlepage}
\frame{\frametitle{Table of contents}\tableofcontents}


\section{Array}
\frame{\frametitle{Basic Array}
    Here we will discuss the most basic example of an array in python.
    \vspace{\baselineskip}

    x = [1,2]
    \vspace{\baselineskip}

    x[0] \# this will return 1
    \vspace{\baselineskip}

    x[1] \# this will return 2
}

\section{Looping arrays}
\frame{\frametitle{Basic iterating loop}
    Now here is how we would iterate through each item in a loop

    x = [1,2]
    \vspace{\baselineskip}
    
    for number in x:
    \vspace{\baselineskip}

    \tab print(number)
}

\section{Looping with range}
\frame{\frametitle{Basic iterating loop with range}
    If we want to iterate over a range of numbers, we can utilize the range function. Which takes in a 
    range(start, stop, step)
    \vspace{\baselineskip}

    for i in range(10):
        print(i)
}

\section{Looping exercise}
\frame{\frametitle{Looping exercise}
    Print a countdown starting from 10, and once it hits zero, print blastoff. 
    \vspace{\baselineskip}

    Also, try to have the program wait inbetween countdowns. Hint, look at the python sleep function in the python docs: \hyperref[Python Docs Time](https://docs.python.org/3/library/time.html) You will need 
    an "import time" at the top of your python file
}

\end{document}
