\documentclass{beamer}
\newcommand\tab[1][1cm]{\hspace*{#1}}
\usepackage{listings}
\begin{document}
\title{Python Chapter 4: Functions}
\author{Ezequiel Torres}
\date{\today}
\frame{\titlepage}
\frame{\frametitle{Table of contents}\tableofcontents}




\section{Functions}
\frame{\frametitle{Functions}
    Here we will discuss the most basic example of a Function. Functions
    are for when you want to repeat code.
    \vspace{\baselineskip}

    def myFunction():
        \vspace{\baselineskip}

        print("Hello from a function")
        \vspace{\baselineskip}

    myFunction() \# Prints "Hello from a function"
}


\subsection{Function Arguments}
\begin{frame}[fragile]
\frametitle{Function Arguments}
\begin{lstlisting}
     def hello(name):
        print("hello " + name) 
     
     hello("Zeak") # Prints "hello zeak"
  \end{lstlisting}
\end{frame}

\section{Recursion}
\frame{\frametitle{Functions}
    One cool thing you can do from a function is call itself, for recusion!
    \vspace{\baselineskip}

    def myFunction():
        \vspace{\baselineskip}

        print("I am a good programmer!")
        \vspace{\baselineskip}
        
        myFunction()
        \vspace{\baselineskip}

    myFunction() \# Prints "I am a good programmer" endlessy!
}

\subsection{Recursion Base Case}
\frame{\frametitle{Recursion Base Case}
    What to do when you want to exit? Base case!
    \vspace{\baselineskip}

    def myFunction(count):
        \vspace{\baselineskip}
        
        if count == 100:
        \vspace{\baselineskip}

            return
        \vspace{\baselineskip}


        print("I am a good programmer!")
        \vspace{\baselineskip}
        
        myFunction()
        \vspace{\baselineskip}

    myFunction(0) \# Prints "I am a good programmer" 100 times
}

\subsection{Recursion Exercise}
\frame{\frametitle{Recursion Exercise}
    Generate the sum of the value of Pascal's Triangle at N. 
}


\subsection{Recursion Exercise Solution}
\begin{frame}[fragile]
\frametitle{Recursion Exercise Solution}
\begin{lstlisting}
    def triangle(n):
    if n == 0:
        return []
    elif n == 1:
        return [[1]]
    else:
        newRow = [1]
        result = triangle(n-1)
        lastRow = result[-1]
        for i in range(len(last_row)-1):
            new_row.append(last_row[i] + last_row[i+1])
        newRow += [1]
        result.append(new_row)
    return result
  \end{lstlisting}
\end{frame}
\end{document}
