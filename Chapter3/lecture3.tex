\documentclass{beamer}
\newcommand\tab[1][1cm]{\hspace*{#1}}
\usepackage{listings}
\begin{document}
\title{Python Chapter 2: Arrays and For Loops}
\author{Ezequiel Torres}
\date{\today}
\frame{\titlepage}
\frame{\frametitle{Table of contents}\tableofcontents}




\section{Dictionary}
\frame{\frametitle{Dictionary}
    Here we will discuss the most basic example of a Dictionary. They are for
    key value pairs. Now in python, dictionaries are ordered meaning you can iterate them easily
    \vspace{\baselineskip}

    x = {"Jim": 1, "Tom": 2} 
    \vspace{\baselineskip}

    x["Jim"] \# this will return a 1
    \vspace{\baselineskip}

    x["Tom"] \# this will return a 2
}


\subsection{Dictionary Removal}
\begin{frame}[fragile]
\frametitle{Dictionary Removal}
  \begin{lstlisting}
      x = {"Jim": 1, "Tom": 2}
      x.remove("Jim")
      if "Jim" not in x:
        print("Jim has left")
  \end{lstlisting}
\end{frame}

\subsection{Dictionary Addition}
\begin{frame}[fragile]
\frametitle{Dictionary Additon}
  \begin{lstlisting}
      x = {"Tom": 2}
      x.add("Jim")
      if "Jim" in x:
        print("Jim is back")
  \end{lstlisting}
\end{frame}

\subsection{Dictionary Looping}
\begin{frame}[fragile]
\frametitle{Dictionary Looping}
  \begin{lstlisting}
      x = {"Jim": 1, "Tom": 2, "Kerrie": 3}
      for p in x:
          print(p) \# prints all keys
      for v in x:
          print(v) \# prints all values
  \end{lstlisting}
\end{frame}

\section{Sets}
\frame{\frametitle{Sets}
    Sets are a lot like dictionaries, but they contain only one key! Great for checking when you have
    visited something.

    mySet = {"apple", "banana", "cherry"}
    \vspace{\baselineskip}
    
    if "apple" in myset:
    \vspace{\baselineskip}
        
        print("There is an apple in myset")
        \vspace{\baselineskip}
}

\end{document}
