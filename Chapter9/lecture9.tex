\documentclass{beamer}
\newcommand\tab[1][1cm]{\hspace*{#1}}
\usepackage{listings}
\begin{document}
\title{Python Chapter 9: Trees and Graphs}
\author{Ezequiel Torres}
\date{\today}
\frame{\titlepage}
\frame{\frametitle{Table of contents}\tableofcontents}

\section{Node}
\subsection{Node in Python}
\frame{\frametitle{Basics of Heaps}
    Nodes in python are very simple. We simply use the node class and reference the node class
    in itself
}


\begin{frame}[fragile]
\subsection{Binary Tree Example}
\frametitle{Binary Tree Example}
\begin{lstlisting}
class Treenode:
    def __init__(self, data):
        self.data = data
        self.left = None
        self.right = None
 
class Tree:
    def __init__(self):
        self.root = None\end{lstlisting}
\end{frame}

\begin{frame}[fragile]
\subsection{Print Binary Tree}
\frametitle{Print Binary Tree}
\begin{lstlisting}
def height(root):
    if root is None:
        return 0
    return max(height(root.left), \
    height(root.right))+1
 
 
def getcol(h):
    if h == 1:
        return 1
    return getcol(h-1) + getcol(h-1) + 1
\end{lstlisting}
\end{frame}

\begin{frame}[fragile]
\subsection{Print Binary Tree Cont.}
\frametitle{Print Binary Tree Cont.}
\begin{lstlisting}
def printTree(M, root, col, row, height):
    if root is None:
        return
    M[row][col] = root.data
    printTree(M, root.left, col-pow(2, height-2), \ 
    row+1, height-1)
    printTree(M, root.right, col+pow(2, height-2),\ 
    row+1, height-1)
\end{lstlisting}
\end{frame}

\begin{frame}[fragile]
\subsection{Print Binary Tree Cont.}
\frametitle{Print Binary Tree Cont.}
\begin{lstlisting}
def TreePrinter():
    h = height(myTree.root)
    col = getcol(h)
    M = [[0 for _ in range(col)] \
    for __ in range(h)]
    printTree(M, myTree.root, col//2, 0, h)
    for i in M:
        for j in i:
            if j == 0:
                print(" ", end=" ")
            else:
                print(j, end=" ")
        print("")
\end{lstlisting}
\end{frame}


\begin{frame}[fragile]
\subsection{Print Binary Tree Cont.}
\frametitle{Print Binary Tree Cont.}
\begin{lstlisting}
myTree = Tree()
myTree.root = Treenode(1)
myTree.root.left = Treenode(2)
myTree.root.right = Treenode(3)
myTree.root.left.left = Treenode(4)
myTree.root.left.right = Treenode(5)
myTree.root.right.left = Treenode(6)
myTree.root.right.right = Treenode(7)
TreePrinter()
\end{lstlisting}
\end{frame}

\begin{frame}[fragile]
\section{Traversal}
\subsection{Preorder}
\frametitle{Preorder}
\begin{lstlisting}
class Node:
    def __init__(self, v):
        self.data = v
        self.left = None
        self.right = None
\end{lstlisting}
\end{frame}

\begin{frame}[fragile]
\subsection{Preorder Cont.}
\frametitle{Preorder Cont.}
\begin{lstlisting}
def printPreorder(node):
    if node is None:
        return

    # Deal with the node
    print(node.data, end=' ')

    # Recur on left subtree
    printPreorder(node.left)

    # Recur on right subtree
    printPreorder(node.right)\end{lstlisting}
\end{frame}


\begin{frame}[fragile]
\subsection{Preorder Cont.}
\frametitle{Preorder Cont.}
\begin{lstlisting}
root = Node(1)
root.left = Node(2)
root.right = Node(3)
root.left.left = Node(4)
root.left.right = Node(5)
root.right.right = Node(6)

# Function call
print("Preorder traversal of binary tree is:")
printPreorder(root)
\end{lstlisting}
\end{frame}

\begin{frame}[fragile]
\subsection{Inorder}
\frametitle{Inorder}
\begin{lstlisting}
# Function to print inorder traversal
def printInorder(node):
    if node is None:
        return
 
    # First recur on left subtree
    printInorder(node.left)
 
    # Now deal with the node
    print(node.data, end=' ')
 
    # Then recur on right subtree
    printInorder(node.right)
\end{lstlisting}
\end{frame}

\begin{frame}[fragile]
\subsection{Inorder Cont.}
\frametitle{Inorder Cont.}
\begin{lstlisting}
root = Node(1)
root.left = Node(2)
root.right = Node(3)
root.left.left = Node(4)
root.left.right = Node(5)
root.right.right = Node(6)

# Function call
print("Inorder traversal of binary tree is:")
printInorder(root)
\end{lstlisting}
\end{frame}

\begin{frame}[fragile]
\subsection{PostOrder}
\frametitle{PostOrder}
\begin{lstlisting}
def printPostorder(node):
    if node == None:
        return
 
    # First recur on left subtree
    printPostorder(node.left)
 
    # Then recur on right subtree
    printPostorder(node.right)
 
    # Now deal with the node
    print(node.data, end=' ')\end{lstlisting}
\end{frame}

\begin{frame}[fragile]
\subsection{PostOrder Cont.}
\frametitle{PostOrder Cont.}
\begin{lstlisting}
root = Node(1)
root.left = Node(2)
root.right = Node(3)
root.left.left = Node(4)
root.left.right = Node(5)
root.right.right = Node(6)

# Function call
print("Postorder traversal of binary tree is:")
printPostorder(root)
\end{lstlisting}
\end{frame}
\end{document}
